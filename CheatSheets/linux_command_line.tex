\documentclass[landscape]{article}

\usepackage[landscape]{geometry}
\usepackage{hyperref}

\geometry{top=.5in,left=.5in,right=.5in,bottom=.5in}

\begin{document}

\begin{tabular*}{10.5in}{|p{2.9in}|p{2.9in}|p{2.9in}|}
    \begin{flushleft}
        \textbf{\large{Basic Command Examples}}
        \rule{2.9in}{.5pt}
        \small
        \begin{description}
            \item[Extract a field from a delimited file:]
                {\verb!cut -d <delim> -f <field_num>!}
            \item[Extract a range characters from a file]
                {\verb!cut -c <start>-<finish>!}
            \item[Find files that were modified today:]
                {\verb!find . -type f -mtime 0!}
            \item[Perform an action on a subset of files:]
                {\verb!find . -type f | head -n 10!
                 \verb!| xargs -I {} <cmd>!}
             \item[Determine if a file is being used:]
                 {\verb!lsof -V <filename>!}
             \item[Grep with line numbers and context:]
                 {\verb!grep -nA <num_lines> <pattern> *!}
             \item[Show all ports in use:]
                 {\verb!netstat -alp!}
             \item[Show port statistics:]
                 {\verb!netstat -s!}
             \item[Character substitution:]
                 {\verb!tr '<from>' '<to>'!}
             \item[Delete characters:]
                 {\verb!tr '<char>' -d!}
             \item[Show processes as a tree:]
                 {\verb!pstree -ap | grep -C <lines> <proc>!}
             \item[Get the current time as a unix timestamp:]
                 {\verb!date +%s!}
             \item[Get the current time in ISO format:]
                 {\verb!date +%F\ %T!}
             \item[Execute cmd with watch and mark changes:]
                 {\verb!watch -d '<cmd>'!}
             \item[Merge two files together (columns):]
                 {\verb!paste <file1> <file2>!}
             \item[Execute a command as a different user:]
                 {\verb!su -c <cmd>!}
        \end{description}
    \end{flushleft}
    &
    \begin{flushleft}
        \textbf{\large{Sed Examples}}
        \rule{2.9in}{.5pt}
        \small
        \begin{description}
            \item[Basic substitution:]
                {\verb!sed -e 's/<old>/<new>/g' <input>!
                 \verb!> <output>!}
            \item[Delete lines:]
                {\verb!sed -e '/<pattern>/d' <input>!}
            \item[Multiple substitutions:]
                {\verb!sed -e!
                 \verb!'s/<old>/<new>/g;s/<old2>/<new2>/g'!
                 \verb!<input>!}
             \item[Modify a file in-place:]
                {\verb!sed -i 's/<old>/<new>/g' <input>!}
        \end{description}
        \textbf{\large{tr Classes}}
        \rule{2.9in}{.5pt}
        \small
        \begin{description}
            \item[alnum]
                {Alphanumeric Characters}
            \item[alpha]
                {Alphabetic Characters}
            \item[blank]
                {Whitespace Characters}
            \item[cntrl]
                {Control Characters}
            \item[digit]
                {Numeric Characters}
            \item[graph]
                {Graphic Characters}
            \item[lower]
                {Lower-Case Alphabetic Characters}
            \item[print]
                {Printable Characters}
            \item[punct]
                {Punctuation Characters}
            \item[space]
                {Space Characters}
            \item[upper]
                {Upper-Case Alphabetic Characters}
            \item[xdigit]
                {Hexadecimal Characters}
        \end{description}
        \textbf{\large{Basic FTP Commands}}
        \rule{2.9in}{.5pt}
        \small
        \begin{description}
            \item[\texttt{ascii}]
                {Set the transfer type to ascii.}
            \item[\texttt{bell}]
                {Sound an alarm when transfer is finished.}
            \item[\texttt{binary}]
                {Set the transfer type to binary.}
            \item[\texttt{cd <directory>}]
                {Change remote directory.}
            \item[\texttt{exit}]
                {Exit the FTP client.}
            \item[\texttt{get}]
                {Download a file from the server.}
            \item[\texttt{lcd}]
                {Change local directory.}
            \item[\texttt{put}]
                {Upload a file to the server.}
            \item[\texttt{tick}]
                {Display the number of bytes transferred.}
        \end{description}
    \end{flushleft}
    &
    \begin{flushleft}
        \textbf{\large{Bash Tricks}}
        \rule{2.9in}{.5pt}
        \small
        \begin{description}
            \item[One-line for loop:]
                {\verb!for file in *; do <cmd>; done!}
            \item[Use the output of one command in another:]
                {\verb!echo 'foo' > `date +%s`.txt!}
            \item[Same, but using pipes:]
                {\verb!diff <(find /dir1) <(find /dir2)!
                 \verb!tar cvf >(gzip -c > dir.tgz) dir!}
            \item[Fix a mistake in the previous command:]
                {\verb!^old^new!}
            \item[Use the last arg. from the last command:]
                {\verb|rm !$|}
            \item[Run command 2 if command 1 succeeds:]
                {\verb!<cmd1> && <cmd23>!}
            \item[Repeat command for every item in braces:]
                {\verb!touch {1..10}.txt ; rm foo.{obj,bin,bar} ;!
                 \verb!cp <file>{,.bak}!}
            \item[Start/end of line:]
                {\verb!Ctrl-A ; Ctrl-E!}
            \item[Follow the execution of your shell script:]
                {\verb|#!/bin/bash -x|}
            \item[Time multiple commands:]
                {\verb!time sh -c '<commands>'!}
        \end{description}
        \textbf{\large{Vim Specific Tips}}
        \rule{2.9in}{.5pt}
        \small
        \begin{description}
            \item[Open a remote file using vim and ssh:]
                {\verb!vim scp://user@host//path/to/file!}
            \item[Open many files that match a pattern:]
                {\verb!args **/*.java!}
            \item[Modify the contents of the matched files:]
                {\verb!argdo <command> | w!}
            \item[Modify the contents of the opened buffers:]
                {\verb!bufdo <command> | w!}
            \item[Autocomplete Word:]
                {\verb!Ctrl-n Ctrl-p!}
            \item[Indent/Outdent (Insert Mode):]
                {\verb!Ctrl-t, Ctrl-d!}
            \item[Paste to command line:]
                {\verb!Ctrl-r <reg>!}
            \item[Replace with regsiter:]
                {\verb!:s/string/\=@<reg>/!}
        \end{description}
    \end{flushleft}
\end{tabular*}

\begin{tabular*}{10.5in}{|p{2.9in}|p{2.9in}|p{2.9in}|}
    \begin{flushleft}
        \textbf{\large{Redirection}/Processes}
        \rule{2.9in}{.5pt}
        \small
        \begin{description}
            \item[Create a named pipe:]
                {\verb!mkfifo <name>!}
            \item[Process Substitution (Ex):]
                {\verb!diff <(cat foo.txt) <(cat bar.txt)!}
            \item[Use piping in find + exec:]
                {\verb!find . -exec sh -c!
                 \verb!'grep "$1" > "$1.out"' -- {} \;!}
            \item[Use piping with xargs:]
                {\verb!<cmd> | xargs -I {} sh -c!
                 \verb!'grep "$1" > "$1.out"' -- {} \;!}
        \end{description}
    \end{flushleft}
    &
    \begin{flushleft}
        \textbf{\large{}SSH Tips}
        \rule{2.9in}{.5pt}
        \small
        \begin{description}
            \item[Copy IDs:]
                {\verb!ssh-copy-id -i <pub_key> user@host!}
        \end{description}
    \end{flushleft}
    &
    \begin{flushleft}
        \textbf{\large{}File/Dir/Disk Tips}
        \rule{2.9in}{.5pt}
        \small
        \begin{description}
            \item[Find largest files/dirs:]
                {\verb!du -hsx * | sort -rh | head -10!}
            \item[Same, using find:]
                {\verb!find <dir> -print '%s %p\n'!
                 \verb!| sort -nr | head -10!}
        \end{description}
    \end{flushleft}
\end{tabular*}

\end{document}
