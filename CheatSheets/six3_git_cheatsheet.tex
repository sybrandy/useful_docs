\documentclass[landscape]{article}

\usepackage[landscape]{geometry}
\usepackage{hyperref}

\geometry{top=.5in,left=.5in,right=.5in,bottom=.5in}

\begin{document}

\begin{tabular*}{10.5in}{|p{2.9in}|p{2.9in}|p{2.9in}|}
    \begin{flushleft}
        \textsc{Git Workflow (Basic)}
        \rule{2.9in}{.5pt}
        \normalsize{Change ``master'' to the name of your main development
        branch if it is not ``master''}
        \small
        \begin{enumerate}
            \item{\verb!git pull!}
            \item{\verb!git checkout -b <TASK-ID> origin/master!}
            \item{Do the following periodically:}
                \begin{itemize}
                    \item{Commit changes (\verb!git add! and \verb!git commit!)}
                    \item{Get new changes (\verb!git pull!)}
                \end{itemize}
            \item{\verb!git checkout master!}
            \item{\verb!git pull!}
            \item{\verb!git merge --no-ff --no-commit <TASK-ID>!}
            \item{\verb!git commit!}
            \item{\verb!git push!}
        \end{enumerate}

        \textsc{Advanced Workflow Changes/Notes}
        \rule{2.9in}{.5pt}
        \small
        \begin{itemize}
            \item{Use \verb!git pull --rebase! to pull down changes without an
                additional merge commit.}
                \begin{itemize}
                    \item{NOTE: Uncommitted changes need to be stashed first.}
                \end{itemize}
            \item{To rebase instead of merge replace \verb!git merge! \ldots with:}
                \begin{enumerate}
                    \item{\verb!git checkout <TASK-ID>!}
                    \item{\verb!git rebase master!}
                    \item{\verb!git checkout master!}
                    \item{\verb!git merge <TASK-ID>!}
                \end{enumerate}
            \item{Use \verb!git commit -v! to show the diff of the changes
                while editing the commit message.}
            \item{Use \verb!git rebase -i master! to rebase your current branch
                onto the newest commits in the master.}
        \end{itemize}
    \end{flushleft}
    &
    \begin{flushleft}
        \textsc{Finding a commit that introduced a bug (Manually)}
        \rule{2.9in}{.5pt}
        \small
        \begin{enumerate}
            \item{\verb!git bisect start!}
            \item{\verb!git bisect bad!}
            \item{\verb!git bisect good <commit_id>!}
            \item{For each commit bisect makes you examine:}
                \begin{itemize}
                    \item{Run tests.}
                    \item{If tests fail: \verb!git bisect bad!}
                    \item{If tests pass: \verb!git bisect good!}
                \end{itemize}
            \item{When done: \verb!git bisect reset!}
        \end{enumerate}
        \textsc{Finding a commit that introduced a bug (Automated)}
        \rule{2.9in}{.5pt}
        \small
        \begin{enumerate}
            \item{\verb!git bisect start HEAD <commit_id>!}
            \item{\verb!git bisect run <test_script>!}
            \item{When done: \verb!git bisect reset!}
        \end{enumerate}
        \normalsize{The test script must return 0 upon success or anything
        from 1-127 excluding 125 for failure.  Exit code 125 means the source
        cannot be tested, therefore that revision is skipped.}
        \textsc{Gotchas/Tips}
        \rule{2.9in}{.5pt}
        \small
        \begin{itemize}
            \item{Never rebase or amend commits that have already been pushed
                to the server}
            \item{Try to keep commits small.  This makes moving changes between
                branches easier}
            \item{Create aliases for complex commands that are used often,
                such as showing the log history as a graph or the merge
                statement in the basic git workflow.}
            \item{Read the command output as it will give you instructions for
                what to do.  For example, the output of ``git status'' will
                tell you how to un-stage a file and more.}
        \end{itemize}
    \end{flushleft}
    &
    \begin{flushleft}
        \textsc{Tags}
        \rule{2.9in}{.5pt}
        \small
        \begin{description}
            \item[Create an annotated tag]
                {\verb!git tag -a <tag_name> -m '<description>'!}
            \item[List current tags:]
                {\verb!git tag -l!}
            \item[Delete a tag:]
                {\verb!git tag -d <tag_name>!}
            \item[Push tags to the server:]
                {\verb!git push origin --tags!}
            \item[Delete remote tags:]
                {\verb!git push origin :refs/tags/<tag>!}
        \end{description}
        \textsc{Additional Information}
        \rule{2.9in}{.5pt}
        \small
        \begin{description}
            \item[Git Book:]{\url{http://git-scm.com/book}}
            \item[Notes about git commit messages:]
                {\url{http://tbaggery.com/2008/04/19/a-note-about-git-commit-messages.html}}
            \item[Some git tips:]
                {\url{http://mislav.uniqpath.com/2010/07/git-tips/}}
            \item[Git log formatting tips:]
                {\url{http://www.jukie.net/bart/blog/pimping-out-git-log}}
            \item[Merge vs. Rebase:]
                {Several comparisons of the two different techniques for
                combining changes from topic branches into the master branch.}
                \begin{itemize}
                    \item{\href{http://lostechies.com/joshuaflanagan/2010/09/03/use-gitk-to-understand-git-merge-and-rebase/}
                          {Use gitk to understand git merge and rebase}}
                \end{itemize}
        \end{description}
        \textsc{Configuration}
        \rule{2.9in}{.5pt}
        \small
        \begin{description}
            \item[Aliases:]
                {\begin{verbatim}
[alias]
    co = commit -v
    me = merge --no-ff --no-commit\end{verbatim}}
            \item[Merge Option:]
                {\begin{verbatim}
[merge]
	conflictstyle = diff3\end{verbatim}}
        \end{description}
    \end{flushleft}
\end{tabular*}

\begin{tabular*}{10.5in}{|p{2.9in}|p{2.9in}|p{2.9in}|}
    \begin{flushleft}
        \textsc{Commits}
        \rule{2.9in}{.5pt}
        \small
        \begin{description}
            \item[Delete last N commits; unstage the changes:]
                {\verb!git reset HEAD~N!}
            \item[Delete last N commits; keep staged changes:]
                {\verb!git reset --soft HEAD~N!}
            \item[Delete last N commits:]
                {\verb!git reset --hard HEAD~N!}
            \item[Modify the last commit:]
                {\verb!git commit --amend!}
            \item[Modify a series of commits:]
                {\verb!git rebase -i <start_rev>!}
                \begin{itemize}
                    \item{Read the git book for more information before trying
                        this command!}
                \end{itemize}
            \item[View the details of a specific commit:]
                {\verb!git show <commit_id>!}
            \item[Copy a commit to the current branch:]
                {\verb!git cherry-pick <commit_id>!}
            \item[Remove a file from every commit:]
                {\verb!git filter-branch --tree-filter!
                 \verb!'rm -f <filename>' HEAD!}
                \begin{itemize}
                    \item{Use this command only when necessary!}
                    \item{Other developers should not be using the tree when
                        this is occurring and download a new clone after the
                        changes have been made.}
                \end{itemize}
            \item[Go back to a previous version of a single file:]
                {\verb!git checkout <version> <filename>!}
        \end{description}
        \textsc{Log/History}
        \rule{2.9in}{.5pt}
        \small
        \begin{description}
            \item[View the log for the current branch:]
                {\verb!git log!}
            \item[View the log as a graph:]
                {\verb!git log --graph --oneline!}
            \item[Git log with graph and more information:]
                 {\verb!git log --graph!
                  \verb!--pretty=format:'\t%Cred%h!
                  \verb!%Cgreen(%ad)%Creset!
                  \verb!|%C(yellow)%d%Creset!
                  \verb!%s %C(bold blue)<%an>%Creset'!
                  \verb!--abbrev-commit --date=short!}
            \item[Show who modified a file:]
                {\verb!git blame!}
        \end{description}
    \end{flushleft}
    &
    \begin{flushleft}
        \textsc{Uncommitted Changes}
        \rule{2.9in}{.5pt}
        \small
        \begin{description}
            \item[Show diff of unstanged changes:]
                {\verb!git diff!}
            \item[Show diff of stanged changes:]
                {\verb!git diff --stage!}
            \item[Stage a file to be committed:]
                {\verb!git add <filename>!}
            \item[Stage files interactively:]
                {\verb!git add -i!}
            \item[Remove deleted files from git:]
                {\verb!git add -u!}
            \item[Unstage a file:]
                {\verb!git reset HEAD <file>!}
            \item[Reset the current working directory:]
                {\verb!git reset --hard!}
                \begin{itemize}
                    \item{WARNING: Unsaved changes will be lost!}
                \end{itemize}
            \item[Discard changes to a single file]
                {\verb!git checkout -- <file>!}
            \item[Stash changes temporarily:]
                {\verb!git stash!}
            \item[Apply stashed changes:]
                {\verb!git stash apply!}
            \item[Clear the stash:]
                {\verb!git stash clear!}
        \end{description}
        \textsc{Branches}
        \rule{2.9in}{.5pt}
        \small
        \begin{description}
            \item[List local branches:]
                {\verb!git branch!}
            \item[Show details about all branches:]
                {\verb!git branch -a!}
            \item[Show which branches have not been merged:]
                {\verb!git branch --no-merged!}
            \item[Delete a local branch:]
                {\verb!git branch -d <branch_name>!}
            \item[Switch to a branch:]
                {\verb!git checkout <branch_name>!}
            \item[Create a branch from a tag:]
                {\verb!git checkout <branch_name> <tag_name>!}
            \item[Checkout a remote branch:]
                {\verb!git checkout --track origin/<br_name>!}
            \item[Push a branch to the server:]
                {\verb!git push origin <br_name>!}
            \item[Rename a branch:]
                {\verb!git branch -m <new_name>!}
        \end{description}
    \end{flushleft}
    &
    \begin{flushleft}
        \textsc{Other}
        \rule{2.9in}{.5pt}
        \small
        \begin{description}
            \item[Open the git gui:]
                {\verb!git gui!}
            \item[Get a list of files that changed:]
                {\verb!git log --name-only --pretty=oneline!
                 \verb!--full-index <start_rev>..<end_rev>!
                 \verb!| grep -vE '^[0-9a-f]{40} '!
                 \verb!| sort | uniq!}
             \item[Create an archive from a git repo:]
                 {\verb!git archive <revision>!}
                 \begin{itemize}
                     \item{Supports the following formats:}
                         \begin{itemize}
                             \item{tar}
                             \item{tar.gz}
                             \item{zip}
                         \end{itemize}
                 \end{itemize}
             \item[Get a ``version id'' for the current commit:]
                 {\verb!git describe --always --tags!}
        \end{description}
        \textsc{Repositories/Remotes}
        \rule{2.9in}{.5pt}
        \small
        \begin{description}
            \item[Clone a new repository:]
                {\verb!git clone <repository_url>!}
            \item[Show information about remotes:]
                {\verb!git remote show origin!}
            \item[Show the URLs for the remotes:]
                {\verb!git remote -v!}
            \item[Add a remote:]
                {\verb!git remote add <name> <url>!}
            \item[Remove a remote:]
                {\verb!git remote rm <name>!}
            \item[Download changes for all branches:]
                {\verb!git fetch!}
                \begin{itemize}
                    \item{Good for reviewing changes before applying them.
                          E.g. \verb!git diff master origin/master!}
                    \item{Will need to do a \verb!git merge! or
                        \verb!git rebase! to apply the changes.}
                \end{itemize}
        \end{description}
        \textsc{File Management}
        \rule{2.9in}{.5pt}
        \small
        \begin{description}
            \item[Delete a file:]
                {\verb!git rm <file>!}
            \item[Move a file:]
                {\verb!git mv <from> <to>!}
            \item[Clean untracked files from current dir:]
                {\verb!git clean -f!}
        \end{description}
    \end{flushleft}
\end{tabular*}

\end{document}
